\section{Evaluation}
\label{sec:evaluation}

\begin{figure}
	\centering
\small
\begin{tabularx}{\columnwidth}{ll}
\toprule
LLVM & Instruction Syntax\\
\midrule
3.5 & load <ty>* <pointer> \\
3.9 & load <ty>, <ty>* <pointer> \\
3.5 & getelementptr <ptr vector> ptrval, <index type> idx \\
3.9 & getelementptr <ty>, <ptr vector> <ptrval>, <index type> <idx> \\
\bottomrule
\end{tabularx}
	\caption{Instructions produced by \constc with syntax changes from LLVM 3.5 to LLVM 3.9.}
	\label{table:llvm-instr-syntax}\floatspace
\end{figure}

\begin{figure*}
    \centering
    \begin{subfigure}[b]{0.49\textwidth}
		\dbox{\begin{minipage}[t]{3.1in}
\small
\begin{Verbatim}
int ssl3_cbc_remove_padding(const SSL *s, 
  SSL3_RECORD *rec, unsigned block_size, 
  unsigned mac_size) \{
 
  unsigned padding_length, good;
  const unsigned overhead = 1 + mac_size;

  if (overhead > rec->length) return 0;
  padding_length = rec->data[rec->length - 1];

  \textcolor{blue}{good = constant_time_ge(}
    \textcolor{blue}{rec->length, padding_length + overhead);}
  \textcolor{blue}{good &= constant_time_ge(}
    \textcolor{blue}{block_size, padding_length + 1);}

  \textcolor{purple}{padding_length = good & (padding_length + 1);}

  rec->length -= padding_length;
  rec->type |= padding_length << 8;

  \textcolor{GaryGreen}{return constant_time_select_int(good, 1, -1);}
\}
\end{Verbatim}
\end{minipage}
}
		\caption{The original constant-time OpenSSL function written in C.\newline}
		\label{fig:openssl-original}
    \end{subfigure}
    \begin{subfigure}[b]{0.49\textwidth}
		\dbox{\begin{minipage}[t]{3.1in}
\small
\begin{Verbatim}
int ssl3_cbc_remove_padding(const SSL *s, 
  SSL3_RECORD *rec, unsigned block_size, 
  unsigned mac_size)\{

  unsigned padding_length;
  const unsigned overhead = 1 + mac_size;

  if (overhead > rec->length) return 0;
  padding_length = rec->data[rec->length - 1];

  \textcolor{blue}{if(rec->length < padding_length + overhead)}
    \textcolor{GaryGreen}{return -1;}
  \textcolor{blue}{if(block_size < padding_length + 1)}
    \textcolor{GaryGreen}{return -1;}

  \textcolor{purple}{padding_length += 1;}

  rec->length -= padding_length;
  rec->type |= padding_length << 8;

  \textcolor{GaryGreen}{return 1;}
\}
\end{Verbatim}
\end{minipage}
}
		\caption{The OpenSSL function expressed more intuitively, without using its constant-time branch-avoiding routines.}
		\label{fig:openssl-intuitive}
    \end{subfigure}
    \caption{OpenSSL function used to evaluate \constc.}\label{fig:openssl}\floatspace
\end{figure*}

\para{Methodology}
In order to evaluate the correctness of \constc we used \emph{ct-verif}\cite{almeida2016}, a contant-time verification tool that leverages the SMACK and Boogie tools to verify LLVM intermediate representation. Since \constc was built on LLVM 3.9 and \emph{ct-verif} requires LLVM 3.5, the IR \constc produced needed to be slightly modified in order to be LLVM 3.5-compliant and hence be verified with \emph{ct-verif}. In particular, the \texttt{load} and \texttt{getelementpr} instructions output by \constc---whose syntax is given in Figure~\ref{table:llvm-instr-syntax}---were modified to be LLVM 3.5-compliant.

\para{Test Code}
OpenSSL's \path{ssl3_cbc_remove_padding} \linebreak function---given in Figure~\ref{fig:openssl-original}---removes padding from a decrypted SSLv3 CBC record by updating the record's length in constant time. It provides an extensive yet concise test case for \constc since it encapsulates many C language features, such as structs, branches, and memory references; and constant-time features such as a notion of public and private data. In \path{ssl3_cbc_remove_padding}, inputs \texttt{rec->length}, \texttt{rec->data}, \path{block_size}, and \path{mac_size} are private while \path{s} and \texttt{rec->type} are public. Note that the branch in Figure~\ref{fig:openssl-original} does not violate the constant-time property because its condition is based on public values.

The intuitive version of this code is given in Figure~\ref{fig:openssl-intuitive}, without OpenSSL's constant-time routines that avoid branching. Routines \path{constant_time_ge} and \path{constant_time_select_int} compute in constant time a greater-than-or-equal-to condition and a conditional selection, respectively. Figure~\ref{fig:openssl-intuitive} is truer to the programmer's original intended semantics of \path{ssl3_cbc_remove_padding}, without muddying its structure in order to be constant-time. The colored lines of code in Figure~\ref{fig:openssl} correspond to code that is modified in this transformation into a more intuitive structure.

As we don't yet have a syntax for \constc, we wrote the altered \path{ssl3_cbc_remove_padding}
directly in \constc AST. After compilation, we verified using \emph{ct-verif} that the generated LLVM IR was constant time.

\para{Limitations}
There are several limitations of \constc that will be resolved in future work (Section~\ref{sec:future}) and these limitations affect how much of \path{ssl3_cbc_remove_padding} was expressable in \constc. First, \constc does not yet support C structs; thus, \path{rec}'s fields were passed in via elements of a \bytearray. Second, \constc does not yet have any notion of public labels, so all data is conservatively assumed to be private; since memory cannot be referenced by a non-constant value, the expression \path{rec->data[rec->length - 1]} could yet be represented without a workaround. Next, \constc currently only supports ``if-then-else'' and not ``if'' statements; an ``if'' statement must be represented as ``if-then-else'' with a dummy ``else'' branch.
As a result, Figure~\ref{fig:openssl-intuitive} was altered slightly in order to be able to be represented in \constc.
