\newcommand{\constc}{\textsc{ConstanC}\@\xspace}
\newcommand{\ccore}{\textsc{ConstCore}\@\xspace}
\newcommand{\bnot}{\raise-.6ex\hbox{\scalebox{1.1}[1]{\texttt{\~}}}\@}
\newcommand{\band}{\scalebox{1.25}[1]{\texttt{\&}}\@\xspace}
\newcommand{\bor}{\texttt{|}\@\xspace}
\newcommand{\bandx}{\,\scalebox{1.25}[1]{\texttt{\&}}\,}
\newcommand{\borx}{\,\texttt{|}\,}


\section{Formalization}
\label{sec:formalization}

We formalize \constc by first proving the constant-time properties of a new
language (\ccore), then showing the transformation from \constc to \ccore.

The language \ccore is a \textsc{While}-like language with highly limited
control flow. Notably, there are no conditional branching instructions, and the
only loop construct is a for loop with static loop bounds. We adopt the
\emph{program-counter security model} from Molnar et al. \cite{pcsecmodel},
keeping an instruction count as part of our small-step semantics (see
\autoref{coresemantics}). We then show that every function in \ccore has the
following property: For every function $f$, there exists an instruction count
$f_\kappa$ such that for any input $x_1,\dots,x_n$ to $f$ and current
instruction count $\kappa$, the instruction count $\kappa'$ when $f$ returns is
exactly $\kappa + f_\kappa$. That is to say, the number of instructions executed
by any function in \ccore does not depend on the inputs to the function.
\todo{program-transcript model instead?}

We then construct a transformation from \constc to \ccore. Since both \ccore and
\constc are side effect-free languages, we simply need to prove that for every
function $f$ in \constc, the corresponding transformed function
$f_{\textnormal{core}}$ returns an equivalent value given equivalent arguments.

We ensure this equivalence by keeping a ``context'', which is a bitmask
representing the control flow the of the function at the current statement. This
bitmask is either high (all 1s) or low (all 0s).  Any variable assignment
~\texttt{x := v}~ in \constc is transformed to the following statement in
\ccore:
\begin{center}
  \texttt{x := ((ctx\bandx rnset)\bandx v)\borx(\bnot (ctx\bandx rnset)\bandx x)}
\end{center}
where \texttt{ctx} is the context bitmask and ~\band, \bor, \bnot\ represent
bitwise \emph{and}, \emph{or}, and \emph{not} operations respectively. The
\texttt{rnset} variable is an additional bitmask used for tracking early
function return and is described below.  With this transformation, variables are
only updated to new values if the context at the time of execution indicates
that the original program control flow would have made it to that statement.

Conditional branches are transformed by executing the statements in both
branches. However, before each block is executed, the context bitmask is updated
with the branch condition. Thus the statement ~\texttt{if bexpr then s1 else
s2}~ is transformed to:
\begin{center}
  \pbox{\textwidth}{
    \texttt{b := bexpr} \\
    \texttt{oldctx := ctx} \\
    \texttt{ctx := oldctx \band b} \\
    \texttt{s1} \\
    \texttt{ctx := oldctx \band (\bnot b)} \\
    \texttt{s2} \\
    \texttt{ctx := oldctx}
  }
\end{center}
where \texttt{b} is a fresh temporary variable for each instance of a branching
statement. This ensures that nested conditionals still function as expected.

Return statements in \constc are dealt with by constructing two additional
variables for every function: $rval$ (initialized low) and $rnset$ (initialized
high). A return statement in \constc is translated to an assignment to $rval$,
gated by the context as above.  Additionally, $rnset$ (``return value not set'')
is updated as follows:
\begin{center}
  \texttt{rnset := rnset \band (\bnot ctx)}
\end{center}
In this way, $rnset$ remains high until a ``return statement'' is executed under
an active control flow path, at which point it is set low and remains low for
the rest of the function. Since all variable assignments are gated with $rnset$
in addition to the context, no further variable assignments will cause updates.

We have one final problem: even if two functions execute the same number of
instructions, they can take different amounts of time \todo{cite}. Our compiler
currently targets Intel x86 assembly, and it is currently unknown \todo{fact
check} which instructions in this architecture are truly constant time. Our
mitigation strategy is to restrict ourselves to assumed ``safe'' instructions,
such as basic arithmetic (except division) and comparison operators, as well as
simple loads, stores, and calls. Thus the constant-time nature of compiled
\ccore is no less secure than the set of chosen instructions.
