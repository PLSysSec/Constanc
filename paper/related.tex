\section{Related Work}
\label{sec:related}


Over the years, we have seen an increase in potential attack vectors for
malicious parties. One such attack vector is side channel attacks --- utlizing
leaked information to undermine the security of applications and systems.
Our related work is split into two main areas: a discussion on side channel
attakcs and mitigations as well as a discussion on constant time
implementations.

There has been a plethora of work done on the prevalence of side channel
attacks. 

Moverover, there has been plenty of work completed on avoiding these
side-channel attacks, mainly through variosu constant time implementations.


The importance of the world web in our daily lives is a well worn
justification for its study. Of particular importance to this work is
the comprehensively studied area of web caching in which researchers
observe that popular websites exist and that storing local copies of
these sites can provide performance gains and bandwidth
reductions\cite{breslau1999web}. There are a myriad of techniques for
performing caching that rely on locality of reference for pages and
objects \cite{rabinovich2002web}. While exploring the long history of
web caching is beyond the scope of the paper, we note several pieces
of work that tease out specific properties of web pages. For example,
researchers have noted that caching can be performed at finer-grained
levels than pervious discussed \cite{Wang:2014:MWM:2663716.2663739}
and others have pointed out that pages consist of extraneous content
that might be suitable for filtering
\cite{Krishnamurthy:2006:CMC:1135777.1135829}. Further, recent studies
in this area explore interesting properties of web pages and objects
for use in specific areas such as the broadband domain
\cite{sundaresan2013measuring} as well as for use by mobile devices
\cite{Qian:2012:WCS:2307636.2307649}.

Several measurement papers have explored the recent trends in
page-level characteristics as well as in redundancy and caching
\cite{ihm2011towards}. A striking finding that shares motivation with
this work is the increased complexity of web pages. Ihm et
al. \cite{butkiewicz2011understanding,ihm2011towards}, for example,
explore metrics for measuring website complexity and characterize
object types as well as their locations, servers, origins. Much of
this complexity and third-party content has been attributed to online
advertising. Several papers explore this space, but most notably
Barford et al.  \cite{Barford:2014:AHA:2566486.2567992} and Guha et
al. \cite{Guha:2010:CMO:1879141.1879152}.

One implication of this complexity is an increase in the opaque nature
of the interactions on the web. Several pieces of work shine light on
this problem at a coarse-grained level, for example, by using DNS to
understand transactions \cite{Bermudez:2012:DRD:2398776.2398819} or by
examining interactions and dependancies between websites
\cite{Pujol:2014:BWT:2663716.2663756}. An additional consequence of
complexity is performance. A new body of work explores how to
understand and improve web page load performance \cite{180330}. A
particularly relevant piece of work in this domain is \cite{194916},
which creates fine-grained dependency graphs in order to help
prioritize object loads.

The potential security and privacy consequences of third-party content
are well explored. Initial work in the drive by download arena
\cite{Provos:2008:YIP:1496711.1496712,Cova:2010:DAD:1772690.1772720,Provos:2007:GBA:1323128.1323132}
note that in addtion to web-page compromise, inclusion of object from
other domains create risk. Malicious advertising, as a particular form
of third party content on a website, has received considerable
attention
\cite{Zarras:2014:DAM:2663716.2663719,Li:2012:KYE:2382196.2382267}. Another
form of third-party content that has been an active area of
measurement is exploring the privacy implications of online tracking
\cite{mayer2012third,roesner2012detecting}.


